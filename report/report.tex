\documentclass[11pt, a4paper]{article}
\usepackage[utf8]{inputenc}
\usepackage{graphicx}
\usepackage{float}
\usepackage{pdfpages}
\usepackage{hyperref}
\usepackage{listings}
\usepackage{color}
\usepackage{courier}
\usepackage{multicol}
\graphicspath{ {./images/} }
\usepackage[margin=2cm]{geometry}
\usepackage{minted}
\usemintedstyle{manni}
\setminted{breaklines, python3, autogobble, linenos, frame=lines, framesep=2mm}
\hypersetup{
  colorlinks   = true, %Colours links instead of ugly boxes
  urlcolor     = red, %Colour for external hyperlinks
  linkcolor    = black, %Colour of internal links
  citecolor   = blue %Colour of citations
}

\begin{document}
    \section{Generating Test Data}
        \subsection{Running the generation script}
        For convenience a script was included to automate the generation of test data called generate\_test\_data.py. Invoke the following command to generate the test data.
        \begin{listing}[H]
            \begin{minted}{bash}
                $ chmod +x generate_test_data.py
                $ ./generate_test_data.py
            \end{minted}
            \caption{Generating the test data}
        \end{listing}
    \paragraph{} A class dedicated to manipulate CSV files was created called csv\_utils. This class also has a method for generating a matrix representing a combination input bits.

    \begin{listing}[H]
        \begin{minted}{python}
            def to_matrix(n: int):
                """
                A helper function that generates a matrix of bit combinations
                :param n: size of the matrix
                :return: a matrix with the input combinations
                """
                def gen(n: int):
                    for i in range(1, 2 ** n - 1):
                        yield '{:0{n}b}'.format(i, n=n)

                matrix = [[0 for i in range(n)]]
                for perm in list(gen(n)):
                    matrix.append([int(s) for s in perm])
                matrix.append([1 for i in range(n)])
                return matrix
        \end{minted}
        \caption{Input bit generation}
    \end{listing}
    
    \paragraph{} The following method is responsible for generating the data set.

    \begin{listing}[H]
        \begin{minted}{python}
            def generate_data_to_csv(matrix_size: int, file_name: str = 'hard_problem', transformation_function=transformation):
                """
                A helper function to aid in generating csv data
                :param transformation_function: The transformation function for generating the output bits
                :param matrix_size: The size of the matrix
                :param file_name: The file name to produce
                :return: Input matrix and output matrix
                """
                input_array = np.asarray(to_matrix(matrix_size))
                output = np.apply_along_axis(transformation_function, 1, input_array)
                data_frame = pd.DataFrame(np.concatenate((input_array, output), axis=1))
                data_frame.to_csv(os.path.join('resources', f'{file_name}.csv'), header=None, index=None)
                return input_array, output
        \end{minted}
        \caption{Feed forward implementation}
    \end{listing}

    \begin{listing}[H]
        \begin{minted}{python}
            def feedforward(self, inp: np.ndarray = None) -> (np.ndarray, np.ndarray):
                """
                A feed forward method that allows the neural net to 'think'.
                :param inp: a numpy array representing the inputs
                :return: a tuple representing the output of the hidden and final output
                """
                if inp is None:
                    inp = self.inputs
                net_h = np.dot(inp, self.wh)
                out_h = self.activation(net_h)
                net_o = np.dot(out_h, self.wo)
                out_o = self.activation(net_o)
                return out_h, out_o
        \end{minted}
        \caption{Feed forward implementation}
    \end{listing}
\end{document}